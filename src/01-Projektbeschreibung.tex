\section{Projektbeschreibung}

Durch das Framework Electron\footnote{\url{https://electronjs.org/}} werden Programme für Desktop-Systeme \textit{(Windows, Linux und macOS)} mit Webtechnologien \textit{(Frontend)} und Node.js\footnote{\url{https://nodejs.org/}} \textit{(Backend)} entwickelt.
Mobile-Systeme \textit{(Android - ChromeOS, AndroidTV / iOS - iPadOS, tvOS)} werden nicht unterstützt.
Um Mobile-Anwendungen mit Webtechnologien zu entwickeln, steht das Capacitor\footnote{\url{https://capacitorjs.com/}}-Framework zur Verfügung, jedoch fehlt dort das Node.js Backend.

Im Rahmen der Diplomarbeit ist geplant, das Capacitor-Framework um ein Node.js Backend zu erweitern, sowie eine direkte Integration in die Electron Plattform-Erweiterung\footnote{\url{https://github.com/capacitor-community/electron}} für Capacitor, um eine einfache und plattformübergreifende App-Entwicklung mit Webtechnologien und Node.js zu ermöglichen, welche ohne zusätzlichen Aufwand auf allen gängigen Betriebssystemen \textit{(Windows, Linux, macOS, Android und iOS)} läuft.
\textit{(Teil 1)}

Des Weiteren ist eine Portierung der BrowserView\footnote{\url{https://electronjs.org/de/docs/latest/api/browser-view}} Funktion von Electron zu Capacitor geplant, sowie eine direkte Integration in die Electron Plattform-Erweiterung für Capacitor, um eine plattformübergreifende Entwicklung weiterhin zu gewährleisten.
\textit{(Teil 2)}

Mithilfe dieser beiden Erweiterungen für das Capacitor-Framework ist geplant, einen plattformübergreifenden PWA\footnote{\url{https://developer.mozilla.org/en-US/docs/Web/Progressive_web_apps}} \textit{(Progressive Web-App)} Wrapper zu erstellen.
Damit Entwickler aus ihrer WebApp, eine \textit{(\enquote{Container}-)}Anwendung mit zusätzlichen Node.js Backend"=Funktionen einfach erstellen können.
Der WebApp-Wrapper soll eine minimale und moderne Benutzeroberfläche sowie eine API enthalten, um die geladene WebApp bzw.\ Webseite zu steuern, die Benutzeroberfläche anzupassen und um zwischen Frontend und Backend zu kommunizieren.
Die Benutzeroberfläche soll bei auftretenden Fehlern auf der Benutzer- oder Server-Seite\textit{, wie beispielsweise bei fehlerhaftem Laden der WebApp,} sowie bei verschiedenen Browser-Ereignissen\textit{, wie beispielsweise während dem Laden der WebApp,} entsprechend benutzerfreundlich reagieren.
\textit{(Teil 3)}
